\documentclass[11pt]{article}

\usepackage{xltxtra}
\usepackage{fontspec}
\setmainfont{TeX Gyre Pagella}

\usepackage[a4paper, top=2cm, bottom=2cm, left=3cm, right=3cm]{geometry}

\usepackage{amsmath}
\usepackage{graphicx}
\usepackage{hyperref}
\usepackage{verbatim}
\usepackage{paralist}
\usepackage{minted}

\usepackage[normalem]{ulem}

% macro for MetaModelElements
\newcommand{\smme}[1]{\uline{\texttt{#1}}}
\newcommand{\tmme}[1]{\texttt{#1}}

\newcommand{\email}[1]{\href{mailto:#1}{\texttt{#1}}}
\newcommand{\myemail}[0]{\email{horn@uni-koblenz.de}}

%% Schusterjungen und Hurenkinder verhindern. Das sind einzelne Zeilen von
%% Absätzen am Anfang oder Ende einer Seite.
\clubpenalty = 10000
\widowpenalty = 10000
\displaywidowpenalty = 10000


\title{The TTC 2013 Flowgraphs Case}
\author{Dipl.-Inform. Tassilo Horn\\
  \myemail\\
  \vspace{0.3cm}\\
  University Koblenz-Landau, Campus Koblenz\\
  Institute for Software Technology\\
  Universitätsstraße 1, 56070 Koblenz, Germany}

\begin{document}

\maketitle

\begin{abstract}
  This case for the Transformation Tool Contest 2013 is about evaluating the
  scope and usability of transformation languages and tools for a set of four
  tasks requiring very different capabilities.  Two tasks deal with typical
  model-to-model transformation problems, there's a model-to-text problem,
  there are two in-place transformation problems, and finally there's a task
  dealing with validation of models resulting from the transformations.

  The tasks build upon each other, but the transformation case project also
  provides all intermediate models, thus making it possible to skip tasks that
  are not suited for a particular tool, or for parallelizing the work among
  members of participating teams.

  All models and metamodels are provided in the EMF/Ecore format.  However,
  user's of other modeling frameworks are also highly encouraged to
  participate.  If requested, export tools can be written to serialize the
  models in other formats such as simple CSV files or GXL.
\end{abstract}

\sloppy

\section{Objective of the Case}
\label{sec:objective}

The objective of this case is to allow participants to demonstrate their
transformation tool's flexibility, i.e., to show its usability for a very
diverse set of tasks requiring different transformation capabilities.  Although
different capabilities are needed, all tasks are connected by their general
topic: analysis and transformation in compiler construction.

All models and metamodels are provided in the EMF/Ecore format.  However,
user's of other modeling frameworks are also highly encouraged to participate.
If requested, the case proponent is willing to help writing an export tool to
other formats such as CSV or GXL \cite{GXL02}.

Task~1 deals with a typical model-to-model transformation problem.  Given an
abstract syntax graph of a Java program conforming to the very detailed
EMFText\footnote{\url{http://www.emftext.org/index.php/EMFText}} JaMoPP
metamodel \cite{jamopp09}, a new model, the structure graph of the original
program, conforming to a much more abstract and simple metamodel has to be
generated.  Embedded in this task is a model-to-text transformation.  In the
JaMoPP model, a simple statement like \verb|int i = a + b;| is expressed as a
whole bunch of interrelated objects.  In the target model, we only want one
single \verb|SimpleStmt|, but its \verb|txt| attribute should be set to the
original Java syntax: \verb|int i = a + b;|.

In task~2, the structure graph resulting from task~1 should be enhanced with
control flow information, i.e., every statement or expression has to be
connected with its possible control flow successors as defined by the Java
semantics \cite{Java7Spec}.  This is an in-place transformation task which is
suited for graph transformation tools but can also be tackled algorithmically.

Task~3 is similar to task~2 from a technical point of view, that is, it is also
an in-place transformation.  Based on the control flow graph resulting from
task~2, data flow information has to be synthesized.  This also requires some
additions to the model-to-model transformation from task~1.  Again, this task
is suited to be tackled with graph transformations or algorithmically.

The context of task~4 is a bit offside the strict transformation context.
Nevertheless, it tackles the important challenge of model validation.  Given
the fact that you as transformation engineer put all your efforts in the
transformation tasks, can you offload testing to developers that have a good
Java knowledge but know nothing about MDE, EMF, or whatever modeling technology
you are using?

Because every task builds upon the results of previous tasks, the intermediate
models are also provided to allow participants to defer or skip tasks not
particulary suited for their tools, or to allow teams for developing solutions
in parallel.  For this reason, there is no distinction between \emph{core} and
\emph{extension} tasks.  To participate, only one arbitrary task has to be
solved, but of course scoring high presumably requires solving more tasks.


\section{Detailed Task Description}
\label{sec:task-descr}

\paragraph{Structure of the Case Project.}

The case project is available on GitHub\footnote{This case's project:
  \url{https://github.com/tsdh/ttc-2013-flowgraphs-case}}. The top-level folder
\verb|ttc-2013-flowgraphs-case| contains several directories.  The directory
\verb|desc| contains the case description your are reading right now.

The \verb|metamodel| directory contains the target metamodel
\verb|FlowGraph.ecore| including PDF images with several views on it focusing
on the specific tasks.  \verb|StructureGraph.pdf| shows the target metamodel
parts relevant for task~1, \verb|ControlFlowGraph.pdf| shows the metamodel
classes important for task~2, and \verb|DataFlowGraph.pdf| shows the metamodel
classes important for task~3.

The \verb|src| folder contains several Java classes.  Every class contains just
one single method.  For every class, one model will be generated as source for
task~1.  If you need more models to test your transformations, simply put a new
Java file in this directory.

The \verb|jamopp| directory contains the JaMoPP-Parser as JAR file that
generates EMF models conforming to the JaMoPP metamodel out of Java source code
files.

Initially, the \verb|models| directory is empty.  Top-level, there are two
shell scripts \verb|gen-models-from-src.sh| (for Unices) and
\verb|gen-models-from-src.bat| (for Windows).  When being run, they use JaMoPP
to parse all Java files in \verb|src| and create corresponding models (file
extension \verb|java.xmi|) in \verb|models|.  Those are valid source models for
task~1.  Additionally, the JaMoPP parser serializes the JaMoPP metamodel next
to the model files.  It is called \verb|java.ecore|\footnote{It'll also create
  a file \texttt{layout.ecore} containing classes that can be used to annotate
  abstract syntax graphs with layout information such as indentation, but this
  is of no importance for this case.} and the source metamodel of task~1.

The \verb|results| folder contains all intermediate and final target models
including visualizations.  You can use them to validate your own results, or
use them as source models for later tasks in case you skip an earlier task.

Finally, the \verb|evaluation| directory contains an OpenDocument spreadsheet
that will be used for scoring the solutions during the evaluation.


\subsection{Task~1: Structure Graph}
\label{sec:task1-structure-graph}

The first task requires writing a model-to-model transformation.  The source
models are the Java abstract syntax graphs conforming to the JaMoPP metamodel
that are created by the \verb$gen-models-from-src.{sh,bat}$ scripts.  The
JaMoPP metamodel covers the complete syntax of Java 7.  However, to restrict
the size of the transformation, the elements actually contained in the source
models is very limited.  For every \verb|*.java| file, the corresponding model
contains one compilation unit containing exactly one class with exactly one
method.  The method may have parameters.  In the method's body, there may be
local variable declarations, arithmetic expressions (only \verb|+|, \verb|-|,
\verb|*|, and \verb|/|), assignments, unary modification expressions
(\verb|i++;| and \verb|i--;|), \verb|return| statements, and blocks.  There may
be \verb|if|-statements and \verb|while|-loops with a boolean expression as
condition.  Statements may be labeled, and \verb|break|/\verb|continue| may be
used with or without target label.  Section~\ref{sec:list-used-jamopp} in the
appendix lists all actually used concrete JaMoPP classes.

The target metamodel of the transformation is depicted in
Figure~\ref{fig:structure-graph-mm}.

\begin{figure}[h!]
  \centering
  \includegraphics[width=\linewidth]{../metamodel/StructureGraph}
  \caption{The target structure graph metamodel}
  \label{fig:structure-graph-mm}
\end{figure}

The metamodel is very similar to the original JaMoPP metamodel from a
structural point of view.  The major difference is that statements and
expressions are represented as one single object instead of being split up any
further.  Another difference is that every \verb|Method| has exactly one
\verb|Exit|.  There is no correspondence in Java, but it's a synthetic element
added in favour of task~2.  No matter how a method is exited, the last object
in a method's control flow graph is this method's \verb|Exit| object.

All metamodel classes extend the abstract \verb|Item| class, even the class
\verb|Expr| although not visible in Figure~\ref{fig:structure-graph-mm} because
an abstract, intermediate class between \verb|Item| and \verb|Expr| is not
displayed for readabily reasons.  \verb|Item| declares a \verb|txt| attribute.
The transformation has to set the value of this attribute to the concrete Java
syntax of the statement or expression, that is, there is a model-to-text
transformation embedded in this model-to-model transformation.  For example,
for a local variable statement with a decimal integer literal as initial value
like \verb|int i = 17;| in the JaMoPP model, a \verb|SimpleStmt| has to be
created and its \verb|txt| attribute has to be set to \verb|"int i = 17;"|.
Exceptions from this rule are all structured statements: the \verb|txt| value
of a \verb|Block| is always \verb|"{...}"|, the value for an \verb|If| is
always \verb|"if"|, the value for a \verb|Loop| is always \verb|"while"|, the
value for a \verb|Method| with name \verb|foo| is always \verb|"foo()"| (no
matter if there are parameters).  The artificial \verb|Exit| objects should
have the value \verb|"Exit"| set.

With the exception of \verb|Break| and \verb|Continue| objects that might refer
to a target \verb|Label|, the structure graphs created by the transformation
are simple trees that reflect the containment hierarchy of the method.

For example, Listing~\ref{lst:sg-ex} shows the method defined in
\verb|Test1.java|.

\begin{listing}
  \begin{minted}{java}
public static void testMethod(int a) {
    int i = a * 2;
    i = i + 19;
    while (i > a) {
        if (a < 1) {
            return;
        } else if (a == 1) {
            break;
        }
        i--;
    }
}
  \end{minted}
  \caption{An example Java method (\texttt{Test1.java})}
  \label{lst:sg-ex}
\end{listing}

The resulting target model is visualized in Figure~\ref{fig:sg-test1} in
Appendix~\ref{sec:result-models} on page~\pageref{fig:sg-test1}.


\subsection{Task~2: Control Flow Graph}
\label{sec:task2-cf-graph}

This task deals with an in-place transformation problem.  The semantics of the
Java programming language should be integrated into the structure graphs
created by the previous transformation.  The task is to perform an
intra-procedural control flow analysis.  Any instruction should be connected to
the instruction following it in the control flow.

Figure~\ref{fig:control-flow-mm} shows the relevant metamodel excerpt.

\begin{figure}[h!]
  \centering
  \includegraphics[width=0.8\linewidth]{../metamodel/ControlFlowGraph}
  \caption{Metamodel classes related to control flow}
  \label{fig:control-flow-mm}
\end{figure}

Simple statements, expressions, the synthetical exits, methods, \verb|return|,
and the jump statements \verb|break| and \verb|continue| extend
\verb|FlowInstr|.  Every flow instruction knows it immediate control flow
predecessors (\verb|cfPrev|) and successors (\verb|cfNext|).  It's those links
the transformation has to synthesize from the structure graph.

Blocks, labels, loops, and if-statements don't participate in the control flow.
Instead, when control flow reaches a block, the first flow instruction in the
block is the control flow successor of the previous flow instruction.  Since
blocks may be nested in other blocks, the \emph{first} flow instruction is
actually the first one reachable by a depth-first search.  These \emph{first}
semantics apply to whole description of this task.

In case of a label, the first flow instruction in the labled statement is the
control flow successor.

In case of loops and if-statements, the successor is their test expression.
This expression has in turn two control flow successors.  If it is a test
expression of a loop, the successors are the first flow instruction in the
loop's body, and the first flow instruction following the loop.  If it is a
test expression of an if-statements, the first successor is the first flow
instruction in then-statement.  If there is an else-statement, its first flow
instruction is the other control flow successor.  Otherwise, the other
successor is the first flow instruction in the statement following the
if-statement.

The control flow successor of a \verb|Method| is its first flow instruction,
and \verb|Return| statements always have the method's \verb|Exit| as control
flow successor.

The most complex control flow rules apply to the jump statements \verb|Break|
and \verb|Continue|.  Without a target label, the control flow successor of a
\verb|Break| is the first flow instruction following the immediately
surrounding loop, and the successor of a \verb|Continue| is the test expression
of the immediately surrounding loop.  With a target label \verb|x|, the control
flow successor of a \verb|Break| is the first flow instruction following the
statement labeled \verb|x|, and the successor of a \verb|Continue| is the
expression of the surrounding loop labeled \verb|x|\footnote{Interestingly,
  using \texttt{break x;} it is possible to jump out of any statement labeled
  \texttt{x}, i.e., in contrast to \texttt{continue}, the labeled statement
  doesn't need to be a loop.}.


The method in Listing~\ref{lst:java-ex-cf} is the method defined in
\verb|Test2.java|.

\begin{listing}
  \begin{minted}{java}
public static void testMethod(int a) {
    int i = a * 2;
    while (i > a) {
        if (a < 1) {
            return;
        } else if (a == 0) {
            continue;
        }
        i++;
    }
}
  \end{minted}
  \caption{An example Java method with complex control flow
    (\texttt{Test2.java})}
  \label{lst:java-ex-cf}
\end{listing}

It's structure graph that is the result of task~1's transformation is the input
to the control flow transformation.  In case you haven't tackled task~1 yet, it
is also included in this project as \verb|results/Test2-StructureGraph.xmi|.
The result control flow graph is visualized in Figure~\ref{fig:cfg-test2} in
Appendix~\ref{sec:result-models} on page~\pageref{fig:cfg-test2}.  The control
flow transformation should create a model equivalent to
\verb|results/Test2-ControlFlowGraph.xmi|.


\subsection{Task~3: Data Flow Graph}
\label{sec:task3-df-graph}

In this task, an intra-procedural data flow analysis should be performed.  This
can be done based on the control flow graph, but currently one important piece
of information is missing from it: for every flow instruction, we need to know
which variables it reads and writes.  Therefore, this task is twofold:

\begin{compactenum}
\item Extend the model-to-model transformation from task~1 so that it also
  creates \verb|Var| objects for local variables and \verb|Param| objects for
  method parameters, and connect each flow instruction to the variables it
  reads and writes.
\item Write a data flow transformation taking the control flow graph resulting
  from applying task~2's transformation on the result of the extended
  java-to-structure-graph transformation that synthesizes data flow links.
\end{compactenum}

The relevant metamodel excerpt is shown in Figure~\ref{fig:data-flow-mm}.

\begin{figure}[h!]
  \centering
  \includegraphics[width=0.6\linewidth]{../metamodel/DataFlowGraph}
  \caption{Metamodel classes related to data flow}
  \label{fig:data-flow-mm}
\end{figure}

\paragraph{Subtask~3.1.}
\label{sec:subtask-3.1}

For every local variable statement in the JaMoPP model, the extended
model-to-model transformation has to create a \verb|Var| object.  It's
\verb|txt| attribute should be set to the variable's name.  Similarly, a
\verb|Param| object has to be created for every method parameter.  Again, the
\verb|txt| attribute should be set to the parameter's name.

Furthermore, every flow instruction should be connected to the variables it
writes (the \verb|def| reference) and to the variables it reads (the \verb|use|
reference).  The method parameters are the \verb|def|-list of their method, the
statement \verb|a = b + c;| has a \verb|def| link to \verb|a|, and two
\verb|use| links to \verb|b| and \verb|c|, and the statement \verb|i++;| both
defines and uses \verb|i|.

Thereafter, task~2's control flow transformation has to be applied to the
result of the extended model-to-model transformation resulting in a structure
graph where every flow instruction knows the variables/parameters it reads and
writes, and which also contains the control flow links between flow
instructions.

Again, the result models of this subtask are contained in the \verb|results|
folder.  For every Java source file \verb|TestX|, there is a model
\verb|TestX-ControlFlowGraph-with-Vars.xmi| plus a PDF visualization of it.  In
case you've skipped this subtask, you can use them to tackle the next subtask.


\paragraph{Subtask~3.2.}
\label{sec:subtask-3.2}

The model resulting from applying the enhanced model-to-model transformation on
the JaMoPP syntax graphs followed by applying the control flow transformation
from task~2 to it is the source model for the data flow transformation to be
developed in this subtask.

It's sole purpose is to synthesize \verb|dfNext| links.  For every flow
instruction $n$, a \verb|dfNext| link has to be created from all nearest
control flow predecessors $m$ that define a variable which is used by $n$.
Formally:

\begin{align*}
  m \rightarrow_{dfNext} n  \iff {} & def(m) \cap use(n) \neq \emptyset\\
  ~\land {} & \exists~Path~m = n_0 \rightarrow_{cfNext} ... \rightarrow_{cfNext} n_k = n:\\
  & \left(def(m) \cap use(n)\right) \setminus \left(\bigcup_{0 < i < k}
    def(n_i)\right) \neq \emptyset
\end{align*}

That is, $n$ uses at least one variable defined by $m$, and there is a control
flow path from $m$ to $n$ in which at least one variable used by $n$ and
defined by $m$ is not redefined by intermediate flow instructions.

There are several ways to tackle this problem.  A simple one is to take the
definition literally, i.e., for every flow instruction search the nearest
control flow predecessors that define a variable used by instruction with
quadratic worst-case effort.  A more efficient and sophisticated algorithm is
described in the dragonbook \cite{Aho:CPTT}, chapter 9.1.  The models resulting
from this task which include control and data flow information are called
\emph{program dependence graphs} (PDG), and they play an important role in the
optimization phase in compilers \cite{Ferrante:1987:PDG:24039.24041}.

The method defined in \verb|Test0.java| is shown in
Listing~\ref{lst:java-ex-df}.  The program dependence graph which this
subtask's transformation has to generate is visualized in
Figure~\ref{fig:pdg-test0} in Appendix~\ref{sec:result-models} on
page~\pageref{fig:pdg-test0}.

\begin{listing}
  \begin{minted}{java}
public int testMethod() {
    int a = 1;
    int b = 2;
    int c = a + b;
    a = c;
    b = a;
    c = a / b;
    b = a - b;
    return b * c;
}
  \end{minted}
  \caption{An example Java method for illustrating data flow
    (\texttt{Test0.java})}
  \label{lst:java-ex-df}
\end{listing}

The result models including visualizations are again available in the
\verb|results| directory.  They are named \verb|TestX-DataFlowGraph.{xmi,pdf}|.
Because the \verb|Var| objects were only needed for computing the data flow
links, they are deleted from the models in order to keep the visualizations
readable.


\subsection{Task~4: Validation}
\label{sec:task4-validation}

The fourth task is no strict transformation task.  Instead, the challenge is
validating the control flow graphs created by task~2, and the program
dependence graphs resulting from task~3.  Concretely, it should be checked if
all \verb|cfNext| and \verb|dfNext| links are set properly.

Since transformation experts have no time for testing their transformations in
the same way as programmers have no time to test their programs, it would be
fabulous if this boring work could be offloaded to other people.  Assume those
don't know anything about MDE in general or more specifically EMF (or whatever
modeling framework you are using), but they are great Java programmers who can
recite every section of the JLS \cite{Java7Spec} from the top of their heads.
So the task is to give them a simple tool that gets a program dependence graph
as input and all control and data flow links in an easy to write textual
syntax.  It should print all missing and all false links, i.e., all links
defined in the textual specification that don't occur in the model, and all
links occuring in the model that are not defined by the specification.

In the example Java programs provided in this case description project, every
statement and expressions occurs at most once in a method, e.g., there's is no
method with two \verb|i++;| statements.  As a result, for all PDGs generated
from them, the \verb|txt| attribute can be used to uniquely identify any
object.  This will be true for any additional methods that might be used for
evaluation.

For example, such a specification for the method shown in
Listing~\ref{lst:java-ex-df} could be written like in
Listing~\ref{lst:example-validation}.  There's no restrictions on the actual
syntax except that it should be easy to write for a Java programmer.

\begin{listing}
\begin{verbatim}
cfNext: "testMethod()"   --> "int a = 1;"
cfNext: "int a = 1;"     --> "int b = 2;"
cfNext: "int b = 2;"     --> "int c = a + b;"
cfNext: "int c = a + b;" --> "a = c;"
cfNext: "a = c;"         --> "b = a;"
cfNext: "b = a;"         --> "c = a / b;"
cfNext: "c = a / b;"     --> "b = a - b;"
cfNext: "b = a - b;"     --> "return b * c;"
cfNext: "return b * c;"  --> "Exit"

dfNext: "int a = 1;"     --> "int c = a + b;"
dfNext: "int b = 2;"     --> "int c = a + b;"
dfNext: "int c = a + b;" --> "a = c;"
dfNext: "a = c;"         --> "b = a;"
dfNext: "a = c;"         --> "c = a / b;"
dfNext: "a = c;"         --> "b = a - b;"
dfNext: "b = a;"         --> "c = a / b;"
dfNext: "b = a;"         --> "b = a - b;"
dfNext: "c = a / b;"     --> "return b * c;"
dfNext: "b = a - b;"     --> "return b * c;"
\end{verbatim}
  \caption{An example textual DSL for validating the program dependence graph
    of the method shown in Listing~\ref{lst:java-ex-df}}
  \label{lst:example-validation}
\end{listing}

The requested tools's job is simple.  It has to load a PDG and to read a
specification such as depicted in Listing~\ref{lst:example-validation}.  For
any \verb|cfNext| or \verb|dfNext| link in the model, it has to check if it is
also defined in the specification.  If not, it has to print a \emph{false-link
  warning} message.  Reversely, it has to check if every link defined in the
specification also occurs in the model.  If not, it has to print a
\emph{missing-link warning}.



\section{Evaluation Criteria}
\label{sec:evaluation-criteria}

In the \verb|evaluation| directory, there is an OpenDocument spreadsheet which
will be used to score all submitted solutions (\verb|evaluation_sheet.ods|).

There are two tables in the spreadsheet.  The large upper one is for
aggregating the individual scores of all voters to compute the overall ranking
between all submitted solutions (\emph{Overall Evaluation Sheet)}.

The small table below is for helping voters to assign a score in the range from
1 (not solved, or very poor) to 5 (excellent) for every task of each submitted
solution (\emph{Per-Task Evaluation Sheet)}.  To score an individual task,
voters should take into account the criteria reflected in that table.  The most
important one is \emph{completeness~\& correctness}, so it is weighted with
50\%.  Another important factor weighted with 30\% is \emph{conciseness~\&
  understandability}.  Lastly, the \emph{efficiency} of the solution is another
criterium that's weighted with the remaining 20\%.  To be able to evaluate
efficiency, further much larger Java methods and corresponding models for
test-driving the solutions will be provided.  Of course, those will use only
the exactly same Java language constructs as the existing examples in order not
to break solutions that used to work fine with the already provided models.

For example, for a given solution's task~1, some voter assigns 5 points to
completeness~\& correctness, 4 point to conciseness \& understandability, and 3
points to efficiency.  The table shows a total weighted score of 4.3 rounded to
4.

The voters score the tasks of all solutions in the same way and write down the
weighted and rounded scores computed by the second table.  Thus, every voter
creates his personal score sheet that might look like
Table~\ref{tab:personal-score-sheet}.

\begin{table}[h!]
  \centering
  \begin{tabular}[h!]{| l | c | c | c | c |}
    \hline
    \textbf{Voter i}& \textbf{Solution 1} & \textbf{Solution 2} & ... & \textbf{Solution n}\\
    \hline
    \textbf{Task~1} & 4 & 3 & ... & 3\\
    \textbf{Task~2} & 3 & 1 & ... & 2\\
    \textbf{Task~3.1} & 4 & 4 & ... & 1\\
    \textbf{Task~3.2} & 5 & 3 & ... & 1\\
    \textbf{Task~4} & 2 & 5 & ... & 2\\
    \hline
  \end{tabular}
  \caption{An example personal score sheet}
  \label{tab:personal-score-sheet}
\end{table}

These personal score sheets of all voters are then conveyed into the overall
evaluation sheet.  For every submitted solution and every task, the number of
voters that assigned a score of $s \in [1, 5]$ is noted down in the cells of
the table.  The table then computes the average score per task for each
solution and weights them to compute a total weighted score.  Task~1 is
weighted with 30\%, Task~2 is weighted with 40\% because its unquestionable the
most complex one, and the tasks 3.1, 3.2, and 4 are weighted with 10\% each.
So the typical model-to-model transformation task~1 and its extension task~3.1
have a weight of 40\% in total, the more graph transformation like tasks 2 and
3.2 have a total weight of 50\%, and the validation task~4 with its 10\% weight
might be the one that tipps the scales.





\bibliography{ttc-2013-flowgraphs-case}
\bibliographystyle{alpha}

\clearpage
\appendix

\section{List of used JaMoPP Classifiers}
\label{sec:list-used-jamopp}

\begin{compactenum}
\item \verb|classifiers.Class|
\item \verb|containers.CompilationUnit|
\item \verb|expressions.AdditiveExpression|
\item \verb|expressions.AssignmentExpression|
\item \verb|expressions.EqualityExpression|
\item \verb|expressions.MultiplicativeExpression|
\item \verb|expressions.RelationExpression|
\item \verb|expressions.SuffixUnaryModificationExpression|
\item \verb|literals.DecimalIntegerLiteral|
\item \verb|members.ClassMethod|
\item \verb|modifiers.Public|
\item \verb|modifiers.Static|
\item \verb|operators.Addition|
\item \verb|operators.Assignment|
\item \verb|operators.Division|
\item \verb|operators.Equal|
\item \verb|operators.GreaterThan|
\item \verb|operators.LessThan|
\item \verb|operators.MinusMinus|
\item \verb|operators.Multiplication|
\item \verb|operators.PlusPlus|
\item \verb|operators.Subtraction|
\item \verb|parameters.OrdinaryParameter|
\item \verb|references.IdentifierReference|
\item \verb|statements.Block|
\item \verb|statements.Break|
\item \verb|statements.Condition|
\item \verb|statements.Continue|
\item \verb|statements.ExpressionStatement|
\item \verb|statements.JumpLabel|
\item \verb|statements.LocalVariableStatement|
\item \verb|statements.Return|
\item \verb|statements.WhileLoop|
\item \verb|types.ClassifierReference|
\item \verb|types.Int|
\item \verb|types.Void|
\item \verb|variables.LocalVariable|
\end{compactenum}

\clearpage
\section{Example Result Models}
\label{sec:result-models}

\begin{figure}[h!]
  \centering
  \includegraphics[width=\linewidth]{../results/Test1-StructureGraph}
  \caption{Structure graph corresponding to \texttt{Test1.java}}
  \label{fig:sg-test1}
\end{figure}

\begin{figure}[h!]
  \centering
  \includegraphics[width=\linewidth]{../results/Test2-ControlFlowGraph}
  \caption{Control flow graph corresponding to \texttt{Test2.java}}
  \label{fig:cfg-test2}
\end{figure}

\begin{figure}[h!]
  \centering
  \includegraphics[height=0.9\textheight]{../results/Test0-DataFlowGraph}
  \caption{Program dependence graph corresponding to \texttt{Test0.java}}
  \label{fig:pdg-test0}
\end{figure}

\end{document}

%%% Local Variables:
%%% mode: LaTeX
%%% TeX-master: t
%%% fill-column: 79
%%% TeX-engine: xetex
%%% TeX-engine-alist: ((xetex "XeTeX" "xetex -shell-escape -output-driver='xdvipdfmx -V 5'" "xelatex -shell-escape -output-driver='xdvipdfmx -V 5'" "xetex"))
%%% End:
